\documentclass{article}

\title{Jug: Software for Reproducible Computation}
\author{Luis Pedro Coelho}

\begin{document}
\section*{Abstract}
As computational pipelines become a bigger part of science, it is important to
ensure that the results are reproducible. All developed software should be able
to be run automatically without any user intervention.

In addition to the value to the community of being able to reproduce an
analysis, reproducible research practices allow for better control over the
project. For example, if necessary parameters to run a pipeline are kept
separately from the code that implements it (perhaps even in the researcher's
mind), this leads to error-prone analysis and opens up the possibility that
when the results are to be writtne up for publication, the researcher will no
longer be able to even completely describe the process that led to them.

For large projects, the use of multiple processors (either in the same machine
or distributed across a cluster) is necessary to obtain results in a useful
timeframe. Furthermore, it is often the case that, as the project evolves, it
becomes necessary to save intermediate results while down-stream analyses are
designed (or redesigned) and implemented. Therefore, having a single point of
entry (often in the form of a single script or a single main programme) for the
computation becomes increasingly difficult.

Jug is a software framework which solves all of these problems in a
simple way. Jug supports caching of intermediate results, distribution of
computation as tasks across a network.

Jug is written in pure Python, is completely cross-platform, and available as
free software under the MIT license. Jug is available from
\url{http://github.com/luispedro/jug}.

\section{Introduction}
\section{Methods}
\section{Example}
\section{Discussion}
\end{document}
